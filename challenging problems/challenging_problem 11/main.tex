\documentclass[journal,12pt,twocolumn]{IEEEtran}

\usepackage{setspace}
\usepackage{gensymb}
\singlespacing
\usepackage[cmex10]{amsmath}

\usepackage{amsthm}

\usepackage{mathrsfs}
\usepackage{txfonts}
\usepackage{stfloats}
\usepackage{bm}
\usepackage{cite}
\usepackage{cases}
\usepackage{subfig}

\usepackage{longtable}
\usepackage{multirow}

\usepackage{enumitem}
\usepackage{mathtools}
\usepackage{steinmetz}
\usepackage{tikz}
\usepackage{circuitikz}
\usepackage{verbatim}
\usepackage{tfrupee}
\usepackage[breaklinks=true]{hyperref}
\usepackage{graphicx}
\usepackage{tkz-euclide}

\usetikzlibrary{calc,math}
\usepackage{listings}
    \usepackage{color}                                            %%
    \usepackage{array}                                            %%
    \usepackage{longtable}                                        %%
    \usepackage{calc}                                             %%
    \usepackage{multirow}                                         %%
    \usepackage{hhline}                                           %%
    \usepackage{ifthen}                                           %%
    \usepackage{lscape}     
\usepackage{multicol}
\usepackage{chngcntr}

\DeclareMathOperator*{\Res}{Res}

\renewcommand\thesection{\arabic{section}}
\renewcommand\thesubsection{\thesection.\arabic{subsection}}
\renewcommand\thesubsubsection{\thesubsection.\arabic{subsubsection}}

\renewcommand\thesectiondis{\arabic{section}}
\renewcommand\thesubsectiondis{\thesectiondis.\arabic{subsection}}
\renewcommand\thesubsubsectiondis{\thesubsectiondis.\arabic{subsubsection}}


\hyphenation{op-tical net-works semi-conduc-tor}
\def\inputGnumericTable{}                                 %%

\lstset{
%language=C,
frame=single, 
breaklines=true,
columns=fullflexible
}
\begin{document}


\newtheorem{theorem}{Theorem}[section]
\newtheorem{problem}{Problem}
\newtheorem{proposition}{Proposition}[section]
\newtheorem{lemma}{Lemma}[section]
\newtheorem{corollary}[theorem]{Corollary}
\newtheorem{example}{Example}[section]
\newtheorem{definition}[problem]{Definition}

\newcommand{\BEQA}{\begin{eqnarray}}
\newcommand{\EEQA}{\end{eqnarray}}
\newcommand{\define}{\stackrel{\triangle}{=}}
\bibliographystyle{IEEEtran}
\raggedbottom
\setlength{\parindent}{0pt}
\providecommand{\mbf}{\mathbf}
\providecommand{\pr}[1]{\ensuremath{\Pr\left(#1\right)}}
\providecommand{\qfunc}[1]{\ensuremath{Q\left(#1\right)}}
\providecommand{\sbrak}[1]{\ensuremath{{}\left[#1\right]}}
\providecommand{\lsbrak}[1]{\ensuremath{{}\left[#1\right.}}
\providecommand{\rsbrak}[1]{\ensuremath{{}\left.#1\right]}}
\providecommand{\brak}[1]{\ensuremath{\left(#1\right)}}
\providecommand{\lbrak}[1]{\ensuremath{\left(#1\right.}}
\providecommand{\rbrak}[1]{\ensuremath{\left.#1\right)}}
\providecommand{\cbrak}[1]{\ensuremath{\left\{#1\right\}}}
\providecommand{\lcbrak}[1]{\ensuremath{\left\{#1\right.}}
\providecommand{\rcbrak}[1]{\ensuremath{\left.#1\right\}}}
\newcommand*{\permcomb}[4][0mu]{{{}^{#3}\mkern#1#2_{#4}}}
\newcommand*{\perm}[1][-3mu]{\permcomb[#1]{P}}
\newcommand*{\comb}[1][-1mu]{\permcomb[#1]{C}}
\theoremstyle{remark}
\newtheorem{rem}{Remark}
\newcommand{\sgn}{\mathop{\mathrm{sgn}}}
\providecommand{\abs}[1]{\left\vert#1\right\vert}
\providecommand{\res}[1]{\Res\displaylimits_{#1}} 
\providecommand{\norm}[1]{\left\lVert#1\right\rVert}
%\providecommand{\norm}[1]{\lVert#1\rVert}
\providecommand{\mtx}[1]{\mathbf{#1}}
\providecommand{\mean}[1]{E\left[ #1 \right]}
\providecommand{\fourier}{\overset{\mathcal{F}}{ \rightleftharpoons}}
%\providecommand{\hilbert}{\overset{\mathcal{H}}{ \rightleftharpoons}}
\providecommand{\system}{\overset{\mathcal{H}}{ \longleftrightarrow}}
	%\newcommand{\solution}[2]{\textbf{Solution:}{#1}}
\newcommand{\solution}{\noindent \textbf{Solution: }}
\newcommand{\cosec}{\,\text{cosec}\,}
\providecommand{\dec}[2]{\ensuremath{\overset{#1}{\underset{#2}{\gtrless}}}}
\newcommand{\myvec}[1]{\ensuremath{\begin{pmatrix}#1\end{pmatrix}}}
\newcommand{\mydet}[1]{\ensuremath{}}
\numberwithin{equation}{subsection}
\makeatletter
\@addtoreset{figure}{problem}
\makeatother
\let\StandardTheFigure\thefigure
\let\vec\mathbf
\renewcommand{\thefigure}{\theproblem}
\def\putbox#1#2#3{\makebox[0in][l]{\makebox[#1][l]{}\raisebox{\baselineskip}[0in][0in]{\raisebox{#2}[0in][0in]{#3}}}}
     \def\rightbox#1{\makebox[0in][r]{#1}}
     \def\centbox#1{\makebox[0in]{#1}}
     \def\topbox#1{\raisebox{-\baselineskip}[0in][0in]{#1}}
     \def\midbox#1{\raisebox{-0.5\baselineskip}[0in][0in]{#1}}
\vspace{3cm}
\title{Challenging Problem 11-part 3}
\author{Annu \\ EE21RESCH01010}
\maketitle
\newpage
\bigskip
\renewcommand{\thefigure}{\theenumi}
\renewcommand{\thetable}{\theenumi}
Download all latex codes from 
\begin{lstlisting}
 https://github.com/annu100/AI5002-Probability-and-Random-variables/tree/main.tex/challenging problems
\end{lstlisting}
\section{Problem}
(UGC/MATH 2018 (June set-a)-Q.106) Let $\{X_i\}_{i \geq 1}$ be a sequence of i.i.d. random variables with $E(X_i)=0$ and $V(X_i)=1$. Which of the following are true?
\vspace{0.2cm}
\begin{enumerate}
    \item $\dfrac{1}{n} \sum_{i=1}^n X_i^2 \to 0$ in probability \vspace{0.2cm}
    \item $\dfrac{1}{n^{3/4}} \sum_{i=1}^n X_i \to 0$ in probability \vspace{0.2cm}
    \item $\dfrac{1}{n^{1/2}} \sum_{i=1}^n X_i \to 0$ in probability \vspace{0.2cm}
    \item $\dfrac{1}{n} \sum_{i=1}^n X_i^2 \to 1$ in probability
\end{enumerate}
\section{Solution-Part 3}
Here $X_1,X_2 \ldots X_n$ are i.i.d random variables and to show the argument of part 3,let us assume these are normal random variables with zero mean and unit variance.\\
$X_i \sim N(0,1)$\\

Let $S_n=X_1+X_2+X_3+ \ldots X_n$\\
Now,$S_n$ converges in distribution a normal random variable ,let us say $U$ with zero mean ,Now if we assume $S_n/\sqrt{n}$ converges in probability at all it must be to $U$.\\
Now according to Cauchy criterion for convergence in probability,which says\\

The sequence $(\frac{S_n}{\sqrt{n}})_{n \ge 1}$ converges in probability if and only if \\
$Pr(|\frac{S_n}{\sqrt{n}}-\frac{S_m}{\sqrt{m}}| \ge \epsilon) \to \text{0 for every}, \epsilon \ge 0,\\ \text{provided m,n} \to \infty$\\

\begin{align}
    Pr(|\frac{S_n}{\sqrt{n}}-\frac{S_m}{\sqrt{m}}| \ge \epsilon)\\
    =Pr(\frac{S_n}{\sqrt{n}}-\frac{S_m}{\sqrt{m}} \ge \epsilon)\\
    +Pr(\frac{S_n}{\sqrt{n}}-\frac{S_m}{\sqrt{m}} \le -\epsilon)\\
    \ge Pr(\frac{S_n}{\sqrt{n}} \ge 2\epsilon ,\frac{S_m}{\sqrt{m}} \le \epsilon) \\
    + Pr(\frac{S_n}{\sqrt{n}} \le -2\epsilon ,\frac{S_m}{\sqrt{m}} \ge -\epsilon) \\
    \ge Pr(\frac{S_n}{\sqrt{n}} \ge 2\epsilon ) \\
    + Pr(\frac{S_n}{\sqrt{n}} \le -2\epsilon )\\
    +Pr(\frac{S_m}{\sqrt{m}} \le \epsilon )\\
    +Pr(\frac{S_m}{\sqrt{m}} \ge -\epsilon ) -2\\
     ( m,n \to \infty ) \to 2(Q(2\epsilon) +Q(-\epsilon)-1),\\
\end{align}
where $Q(x)= \int_{x}^{\infty} N(0,1) dx$\\
So,for any $$\epsilon \ge 0$ ,terms inside brackets of $(Q(2\epsilon) +Q(-\epsilon)-1)$ is positive.\\
And so, $\dfrac{1}{n^{1/2}} \sum_{i=1}^n X_i \to 0$ in probability is wrong.\\
This third option is not correct.
\end{document}