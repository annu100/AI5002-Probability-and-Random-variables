\documentclass[journel,12pt,twocoloums]{IEEEtran}

\title{Challenging problems}
\author{Annu-EE21RESCH01010}
\date{13 January 2020}

\usepackage{amsthm}
\usepackage{graphicx}
\usepackage{mathrsfs}
\usepackage{txfonts}
\usepackage{stfloats}
\usepackage{pgfplots}
\usepackage{cite}
\usepackage{cases}
\usepackage{mathtools}
\usepackage{caption}
\usepackage{enumerate}	
\usepackage{enumitem}
\usepackage{amsmath}
\usepackage[utf8]{inputenc}
\usepackage[english]{babel}
\usepackage{multicol}
%\usepackage{xtab}
\usepackage{longtable}
\usepackage{multirow}
%\usepackage{algorithm}
%\usepackage{algpseudocode}
\usepackage{array,multirow}
\usepackage{enumitem}
\usepackage{mathtools}
\usepackage{gensymb}
\usepackage{hyperref}
%\usepackage[framemethod=tikz]{mdframed}
\usepackage{listings}
    %\usepackage[latin1]{inputenc}                                %%
    \usepackage{color}                                            %%
    \usepackage{array}                                            %%
    \usepackage{longtable}                                        %%
    \usepackage{calc}                                             %%
    \usepackage{multirow}                                         %%
    \usepackage{hhline}                                           %%
    \usepackage{ifthen}                                           %%
  \providecommand{\nCr}[2]{\,^{#1}C_{#2}}
  \providecommand{\nPr}[2]{\,^{#1}P_{#2}}
  \lstset{
%language=C,
frame=single, 
breaklines=true,
columns=fullflexible
}

 \begin{document}
 \maketitle
\textbf{Download latex code from here-}\\
\begin{lstlisting}
 https://github.com/annu100/AI5002-Probability-and-Random-variables/tree/main.tex/challenging problems
 \end{lstlisting}

 \section{Challenging problem 9}

Two points are chosen on a line of unit length.Find probability that each of 3 line segements have length greater than 1/4 is...........

\section{SOLUTIONS}

\begin{flushleft}
Imagine choosing one point P1, and then a second point P2. We assume that "at random" means here that the distributions of X and Y are uniform on [0,1] and that P1 and P2 are independent.

We want the probability that 1/4 $\ge$ P1 $\le$ 3/4 and 1/4 $\ge$ P2 $\le$ 3/4 and subtraction of P1 and P2 $\ge$ 1/4$.

Draw the usual square. Draw the line x=$\frac{1}{4}$, x=$\frac{3}{4}$, y=$\frac{1}{4}$, y=$\frac{3}{4}$. By Looking at the K square bounded by these lines.\\

Drawing the two lines subtraction of P1 and P2 is $\pm$ 1/4.\\
We want to find the probability that (P1,P2). lands in the part of K that is not between these two lines. That consists of two isosceles right-angled triangles.
\\
Each of these triangles has legs 14, so their combined area is $$\frac{1}{16}$.




\end{flushleft}



\end{document}

        

