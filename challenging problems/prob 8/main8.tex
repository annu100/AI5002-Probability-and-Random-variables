\documentclass[journel,12pt,twocoloums]{IEEEtran}

\title{Challenging problem 8}
\author{Annu-EE21RESCH01010}
\date{13 January 2020}

\usepackage{amsthm}
\usepackage{graphicx}
\usepackage{mathrsfs}
\usepackage{txfonts}
\usepackage{stfloats}
\usepackage{pgfplots}
\usepackage{cite}
\usepackage{cases}
\usepackage{mathtools}
\usepackage{caption}
\usepackage{enumerate}	
\usepackage{enumitem}
\usepackage{amsmath}
\usepackage[utf8]{inputenc}
\usepackage[english]{babel}
\usepackage{multicol}
%\usepackage{xtab}
\usepackage{longtable}
\usepackage{multirow}
%\usepackage{algorithm}
%\usepackage{algpseudocode}
\usepackage{array,multirow}
\usepackage{enumitem}
\usepackage{mathtools}
\usepackage{gensymb}
\usepackage{hyperref}
%\usepackage[framemethod=tikz]{mdframed}
\usepackage{listings}
    %\usepackage[latin1]{inputenc}                                %%
    \usepackage{color}                                            %%
    \usepackage{array}                                            %%
    \usepackage{longtable}                                        %%
    \usepackage{calc}                                             %%
    \usepackage{multirow}                                         %%
    \usepackage{hhline}                                           %%
    \usepackage{ifthen}    %%
    \newcommand{\sgn}{\mathop{\mathrm{sgn}}}
\providecommand{\abs}[1]{\vert#1\vert}
\providecommand{\res}[1]{\Res\displaylimits_{#1}} 
\providecommand{\norm}[1]{\lVert#1\rVert}
%\providecommand{\norm}[1]{\lVert#1\rVert}
\providecommand{\mtx}[1]{\mathbf{#1}}
\providecommand{\mean}[1]{E[ #1 ]}
  \providecommand{\nCr}[2]{\,^{#1}C_{#2}}
  \providecommand{\nPr}[2]{\,^{#1}P_{#2}}
  \lstset{
%language=C,
frame=single, 
breaklines=true,
columns=fullflexible
}

 \begin{document}
 \maketitle
\textbf{Download latex code from here-}\\
\begin{lstlisting}
 https://github.com/annu100/AI5002-Probability-and-Random-variables/tree/main.tex/challenging problems
 \end{lstlisting}

 \section{Challenging problem 8}


Let $X_1,X_2,.....,X_n$ be independent Poisson random variables with $\mean{X_i}=\mu_i$.Find the conditional distribution of $X_1,...,X_n\biggr\vert\sum_{i=1}^{n}X_i=y$

\section{SOLUTIONS}

\begin{flushleft}
The random variable $X_i$ is distributed as Poisson if the density of $X_i$ is given by\\
$\mean{X_i}=\mu_i$\\
Also,
$\Var{X_i}=\mu$\\
$f(x_i:\mu_i)$=
\begin{cases}
\frac{\brak{e^{-\mu_i}} \times \mu ^{x_i}}{x_i !} & x_i \ge 0\\
0 & otherwise
\end{cases}
Since $X_1,X_2,.....,X_n$ be independent Poisson random variables-\\
Likelihood function L is required to be calculated and it is given by\\
\begin{align}
 L&=\frac{\brak{e^{-\mu_1}} \times \mu ^{x_1}}{x_1 !} \times \frac{\brak{e^{-\mu_2}} \times \mu ^{x_2}}{x_2 !} 
 \cdot \cdot \cdot \frac{\brak{e^{-\mu_n}} \times \mu ^{x_n}}{x_n !} \\   
  &= \frac{\brak{e^{-\mu_n}} \times \mu ^{\sum_{i=1}^{n} x_i}}{\prod_{i=1}^{n}x_i !}\\
 \end{align}
So,above case is only defined when\\
$\sum_{i=1}^{n}X_i=y$ for some y\\
otherwise L=0.\\
Here L is the conditional distribution of $X_1,...,X_n\biggr\vert\sum_{i=1}^{n}X_i=y$\\
Also sum of two independent Possion R.V is also R,V with mean as sum of mean of individual random variables which provides for the proof of above expression for L.


\end{flushleft}



\end{document}

        

