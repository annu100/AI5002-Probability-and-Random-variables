\documentclass[journel,12pt,twocoloums]{IEEEtran}

\title{Assignment 12-Probability and Random Variable}
\author{Annu-EE21RESCH01010}

\date{13 January 2020}

\usepackage{amsthm}
\usepackage{graphicx}
\usepackage{mathrsfs}
\usepackage{txfonts}
\usepackage{stfloats}
\usepackage{pgfplots}
\usepackage{cite}
\usepackage{cases}
\usepackage{mathtools}
\usepackage{caption}
\usepackage{enumerate}	
\usepackage{enumitem}
\usepackage{amsmath}
\usepackage[utf8]{inputenc}
\usepackage[english]{babel}
\usepackage{multicol}
%\usepackage{xtab}
\usepackage{longtable}
\usepackage{multirow}
%\usepackage{algorithm}
%\usepackage{algpseudocode}
\usepackage{array,multirow}
\usepackage{enumitem}
\usepackage{mathtools}
\usepackage{gensymb}
\usepackage{hyperref}
%\usepackage[framemethod=tikz]{mdframed}
\usepackage{listings}
    %\usepackage[latin1]{inputenc}                                %%
    \usepackage{color}                                            %%
    \usepackage{array}                                            %%
    \usepackage{longtable}                                        %%
    \usepackage{calc}                                             %%
    \usepackage{multirow}                                         %%
    \usepackage{hhline}                                           %%
    \usepackage{ifthen}                                           %%
  \providecommand{\nCr}[2]{\,^{#1}C_{#2}}
  \providecommand{\nPr}[2]{\,^{#1}P_{#2}}
  \lstset{
%language=C,
frame=single, 
breaklines=true,
columns=fullflexible
}

 \begin{document}
 \maketitle
\textbf{Download latex code from here-}\\
\begin{lstlisting}
 https://github.com/annu100/AI5002-Probability-and-Random-variables/tree/main.tex/ASSIGNMENT_12
 \end{lstlisting}
\textbf{Download python code from here-}\\
\begin{lstlisting}
 https://github.com/annu100/AI5002-Probability-and-Random-variables/tree/main.py/ASSIGNMENT_12
 \end{lstlisting}
 \section{Ugc-Net june 2019-51}

Consider a Markov Chain with state space {0,1,2,3,4} and transition matrix-\\

\begin{math}
P=\left[
\begin{array}{c c c c c}
    1 & 0 & 0 & 0 & 0\\
    \frac{1}{3} & \frac{1}{3}  & \frac{1}{3}  & 0 & 0\\
    0 & \frac{1}{3}  & \frac{1}{3}  & \frac{1}{3} & 0\\
    0 & 0 & \frac{1}{3} & \frac{1}{3}  & \frac{1}{3}  \\
    0 & 0 & 0 & 0 & 1
\end{array}\right]
\end{math}
Then $\lim_{n \to \infty } {Pr_{23}}^n$ equals


\section{SOLUTIONS}
To understand this,let us first understand Discrete Time Markov Chain(DTMC).\\
A stochastic process along with markov property makes a Discrete Time Markov Chain.\\
$S_j$ here represents a state on our statespace.
$Pr(X_{n+1}=S_j|X_n,X_{n-1},\ldots X_0)=Pr(X_{n+1}=S_j|X_n).$\\
Further,the probabilistics are often fixed in time.\\
Now,Time Homogeneous property/stationary property of MC is-\\
\begin{align*}
  Pr(X_{n+1}=S_j|X_n) &= Pr(X_1=S_j | X_0)\\
                      &= Pr(X_2=S_j | X_1)\\
                      &= Pr(X_3=S_j | X_2)\\
                      &= Pr(X_4=S_j | X_3) 
\end{align*}
So.time homogeneous or stationary property means our one step transition probability are independent of time (n).\\

Now consider the case when n tends to $\infty$.\\
As sum of all elements in a row of probability transition matrix.So,\\
\begin{math}
Pr(X_{n+1}=S_j|X_n=S_i)=Pr_{ij}\\

\sum_{j=1}^{N} Pr(X_{n+1}=S_j|X_n=S_i)=1\\

\sum_{j=1}^{N} P{ij}=1\\
\end{math}
\\
\textbf{Limiting behaviour of discrete time Markov Chain is}\\

\begin{math}
Pr(X_n=S_j) \text{as n tends to $\infty$ ,we need to find -}\\
\text{let}  \pi_j \text{be stationary distribution of MC i.e }\\
\pi_j =Pr(X_n=S_j|X_0=S_1)
\end{math}
\begin{align*}
\lim_{n \to \infty}Pr(X_n=S_j) 
 &= \lim_{n \to \infty} \sum_{i} Pr(X_n=S_j |X_0=S_i) \times Pr(X_0=S_i)\\
 &=\sum_{i} \lim_{n \to \infty} Pr(X_n=S_j |X_0=S_i) \times Pr(X_0=S_i)\\
 &= \sum_{i} \pi_j \times Pr(X_0=S_i)\\
 &=\sum \pi_j \times Pr(X_0=S_i)\\
 &=\pi_j \times \sum_{i} Pr(X_0=S_i)\\
 &=\pi_j
\end{align*}

$\lim_{n \to \infty}Pr(X_n=S_j) =\pi_j$ irrespective of initial distribution i.e $Pr(X_0=S_j)$. \\
This is limiting behaviour of DTMC\\

\textbf{Stationary distribution}\\
starting with initial distribution for inital state $X_0$\\
\begin{math}
\left[
\begin{array}{c}
    \pi_1\\
    \pi_2\\
    \pi_3\\
    \vdots\\
    \pi_N
\end{array}\right]
\end{math}
\\
$\pi_i=Pr(X_0=S_i)$\\
\text{Now for stationary distribution- } 
$\mathbf{\pi}^\intercal \mathbf{P}=\mathbf{\pi}$

where $\pi$ is the stationary distribution.\\
It is left eigen vector of probability ransition matrix $\mathbf{P}$ or eigen vector of $\mathbf{P}^ \intercal$\\
let $\mathbf{A}=\mathbf{P}^\intercal$
so,$|A-\lambda I|=0$ for eigen value calculation.\\
and Eigen vector lies in null space of matrix $A-\lambda I$\\
As it is 5*5 matrix,using python code for calculation of eigen values and eigen vector ,we get
Eigen values are-\\
$\lambda$1=1 , $\lambda$2=1, $\lambda$=0.804, $\lambda$4=-0.138, $\lambda$5=0.333\\
Possible eigenvectors are: \\
eigenvector 1:\\
\begin{math}
\left[
\begin{array}{c}
     1.0\\
     0.\\
     -5.445260\\
     1.4340\\
      -3.16
\end{array}\right]
\end{math}
\\
 eigenvector 2:\\
\begin{math}
\left[
\begin{array}{c}
      0\\
      0\\
      4.5\\
      6.92\\
      -5.26
\end{array}\right]
\end{math}
\\
eigenvector3:\\
\begin{math}
\left[
\begin{array}{c}
      0\\
      0\\
      3.189\\
      -4.89\\
      6.32
\end{array}\right]
\end{math}
\\
eigenvector4:\\
\begin{math}
\left[
\begin{array}{c}
      0\\
      0\\
      3.189\\
      -4.89\\
      -6.32
\end{array}\right]
\end{math}
\\
Now $\lim_{n \to \infty } {Pr_{n}}^n$ approaches a matrix which has structure that all rows of matrix are identical and each row will give limiting distribution and this limiting distribution corresponds to stationary distribution,\\

So,from possible eigen vectors third value will be $\lim_{n \to \infty } {Pr_{23}}^n$.
\\
i.e.1.43,-4.89 or 6.92.\\ (according to my eigen vector obtained as for one eigen value ,eigen vector is not unique and we can get infinite eigen vectors).


\end{document}

        

