\documentclass[journel,12pt,twocoloums]{IEEEtran}

\title{Assignment 14-Probability and Random Variable}
\author{Annu-EE21RESCH01010}

\date{13 January 2020}

\usepackage{amsthm}
\usepackage{graphicx}
\usepackage{mathrsfs}
\usepackage{txfonts}
\usepackage{stfloats}
\usepackage{pgfplots}
\usepackage{cite}
\usepackage{cases}
\usepackage{mathtools}
\usepackage{caption}
\usepackage{enumerate}	
\usepackage{enumitem}
\usepackage{amsmath}
\usepackage[utf8]{inputenc}
\usepackage[english]{babel}
\usepackage{multicol}
%\usepackage{xtab}
\usepackage{longtable}
\usepackage{multirow}
%\usepackage{algorithm}
%\usepackage{algpseudocode}
\usepackage{array,multirow}
\usepackage{enumitem}
\usepackage{mathtools}
\usepackage{gensymb}
\usepackage{hyperref}
%\usepackage[framemethod=tikz]{mdframed}
\usepackage{listings}
    %\usepackage[latin1]{inputenc}                                %%
    \usepackage{color}                                            %%
    \usepackage{array}                                            %%
    \usepackage{longtable}                                        %%
    \usepackage{calc}                                             %%
    \usepackage{multirow}                                         %%
    \usepackage{hhline}                                           %%
    \usepackage{ifthen}                                           %%
  \providecommand{\nCr}[2]{\,^{#1}C_{#2}}
  \providecommand{\nPr}[2]{\,^{#1}P_{#2}}
  \lstset{
%language=C,
frame=single, 
breaklines=true,
columns=fullflexible
}
\usepackage{tikz}
\usetikzlibrary{shapes,arrows,positioning}
 \begin{document}
 \maketitle
\textbf{Download latex code from here-}\\
\begin{lstlisting}
 https://github.com/annu100/AI5002-Probability-and-Random-variables/tree/main.tex/ASSIGNMENT_114
 \end{lstlisting}

 \section{Gate-24 Solution}
\begin{flushleft}
A fair coin is tossed till a head appears for the
first time. The probability that the number of
requried tosses is odd,is ..........

\section{SOLUTIONS}
As we know\\
For odd no of tosses\\

We can get number of tosses like this\\

1, 3 ,5,7.........\\
Then the probability for getting head for the first time is-\\
\begin{align}
Pr(\text{Head 1st time})=(1/2)^1+ (1/2)^3 +(1/2)^5 +(1/2)^7 \ldots 
\end{align}

As we can see this is decreasing $G.P$ series

So sum ut upto infinity.\\
Sum is given by-\\
$S=\frac{a}{(1-r)}$
where.
a is first term of the series=$\frac{1}{2}$

r is common. Ratio=$\frac{1}{4}$
\end{flushleft}
\begin{align}
    Pr(\text{Head 1st time})&=\frac{1/2}{(1-1/4)}
                     &=\frac{2}{3}
\end{align}

So answer is 0.50/0.75

= 2/3
\subsection{MARKOV CHAIN APPROACH}
\textbf{Probabilities at nth toss}
Probabilities at zero toss- $P_0=(1,0,0)$\\
Probabilities after one toss-$P_0=(0.5,0.5,0)$
Each state has the vector of probabilities for going to another state -\\
collecting it together to form matrix\\
\begin{math}
M_3=\left[
\begin{array}{ccc}
     0.5 & 0.5 & 0  \\
     0.5 & 0 & 0.5 \\
     0 & 0 & 1
\end{array}
\right]
\end{math}
\\
\\
\begin{tikzpicture}[font=\sffamily]

        % Setup the style for the states
        \tikzset{node style/.style={state, 
                                    minimum width=2cm,
                                    line width=1mm,
                                    fill=gray!20!white}}

       % Draw the states
        \node[node style] at (0, 0)     (C1)     {C1};
        \node[node style] at (6, 0)     (C2)     {C2};
        \node[node style] at (3, -5.196) (C3) {C3};

        % Connect the states with arrows
        \draw[every loop,
              auto=right,
              line width=1mm,
              >=latex,
              draw=orange,
              fill=orange]
            (C1)     edge[bend right=20]            node {0} (C3)
            (C1)     edge[bend right=20]            node {0.5} (C2)
            (C1)     edge[loop above]            node {0.5} (C1)
            (C2)     edge[loop above]            node {0} (C2)
            (C2)     edge[bend right=20, auto=left] node {0.5} (C3)
            (C2)     edge[bend right=20, auto=left] node {0.5} (C1)
            (C3) edge[bend right=20]            node {0} (C2)
            (C3) edge[bend right=20, auto=left] node {0} (C1)
            (C3)     edge[loop below]            node {1} (C3);
    \end{tikzpicture}
 Note that:Above drawn markov chain is 3 states first order time homogeneous markov chain as transition probabilities are depending only on last one state.\\
 Note that:Above drawn markov chain is 3 states first order time homogeneous markov chain as transition probabilities are depending only on last one state.\\
 
We can get $P_1$ by multiplying $P_0$ with $M_3$.\\
$P_1=P_0 \times M_3$\\
Similarly to get $P_2$ ,we can multiply $P_1$ with $M_3$\\
$P_2=P_1 \times M_3= P_0 \times M_3^2$\\
So,in general $P_n=P_0 \times M_3^n$\\
here,n is number of tosses.\\
but n should be odd for our problem.\\
It is calculated for upto 100 tosses in python program.\\
\end{document}

        

