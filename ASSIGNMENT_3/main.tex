\documentclass[journel,12pt,twocoloums]{IEEEtran}

\title{Assignment 3 -Probability and Random Variable}
\author{Annu-EE21RESCH01010}
\date{13 January 2020}

\usepackage{amsthm}
\usepackage{graphicx}
\usepackage{mathrsfs}
\usepackage{txfonts}
\usepackage{stfloats}
\usepackage{pgfplots}
\usepackage{cite}
\usepackage{cases}
\usepackage{mathtools}
\usepackage{caption}
\usepackage{enumerate}	
\usepackage{enumitem}
\usepackage{amsmath}
\usepackage[utf8]{inputenc}
\usepackage[english]{babel}
\usepackage{multicol}
%\usepackage{xtab}
\usepackage{longtable}
\usepackage{multirow}
%\usepackage{algorithm}
%\usepackage{algpseudocode}
\usepackage{enumitem}
\usepackage{mathtools}
\usepackage{gensymb}
\usepackage{hyperref}
%\usepackage[framemethod=tikz]{mdframed}
\usepackage{listings}
    %\usepackage[latin1]{inputenc}                                 %%
    \usepackage{color}                                            %%
    \usepackage{array}                                            %%
    \usepackage{longtable}                                        %%
    \usepackage{calc}                                             %%
    \usepackage{multirow}                                         %%
    \usepackage{hhline}                                           %%
    \usepackage{ifthen}                                         %%
  \providecommand{\nCr}[2]{\,^{#1}C_{#2}}
  \providecommand{\nPr}[2]{\,^{#1}P_{#2}}
  \lstset{
%language=C,
frame=single, 
breaklines=true,
columns=fullflexible
}

 \begin{document}
 \maketitle
 \textbf{Download Python code from here}\\
\begin{lstlisting}
 https://github.com/annu100/AI5002-Probability-and-Random-variables/blob/main/ASSIGNMENT_3/Assignment_3_Bayes.py
 \end{lstlisting}
\textbf{Download latex code from here-}\\
\begin{lstlisting}
 https://github.com/annu100/AI5002-Probability-and-Random-variables/blob/main/ASSIGNMENT_3/main.tex
 \end{lstlisting}
 \section{Problem Statement-Problem 2.10}
Bag I contains 3 red and 4 black balls and
Bag II contains 4 red and 5 black balls. One
ball is transferred from Bag I to Bag II and
then a ball is drawn from Bag II. The ball
so drawn is found to be red in colour. Find
the probability that the transferred ball is black.
\section{Solutions}
Bag 1 contains 3 red and 4 black balls.\\
Bag 2 contains 4 red and 5 black balls.\\

let  C1: Event of transferring black ball from bag 1 to 2\\
let  C1: Event of transferring red ball from bag 1 to 2\\
let A : Event that the ball drawn from 2 is red after the transfer of a ball from bag 1 to bag 2.\\
\begin{align*}
Pr(C1)= \frac{4}{7}\\
Pr(C2)=\frac{3}{7}\\
Pr(A/C1)=\frac{4}{10}=\frac{2}{5}\\
Pr(A/C2)=\frac{5}{10}=\frac{1}{2}\\
\text{From Baye's theoram} \\
Pr(\text{Drawn ball is red})&=P(A)\\
                     &=Pr(\frac{A}{C1})\times   Pr(C1)+Pr{C2}\times Pr(\frac{A}{C2})\\
                     &=\frac{4}{10}\times \frac{4}{7}+\frac{5}{10}\times \frac{3}{7}\\
                     &=\frac{16+15}{70}\\
                     &=\frac{31}{70}
\end{align*}


The probability that the transferred ball is black\\
It is equal to conditional probability of C1 when event A has already happened\\
The desired probability is given by\\
\begin{align*}
Pr(\frac{C1}{A})&=\frac{Pr(C1 \cap A)}{Pr(A)}\\
        &=\frac{Pr(\frac{A}{C1})Pr(C1)}{Pr(A)}\\
        &=\frac{\frac{4}{10} \times \frac{4}{7}}{\frac{31}{70}}\\
        &=\frac{16}{31}
\end{align*}

Hence the desired probability is $$\frac{16}{31}=0.516$$
            
\end{document}
        


        

